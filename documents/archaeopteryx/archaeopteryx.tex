%%%%%%%%%%%%%%%%%%%%%%%%%%%%%%%%%%%%%%%%%%%%%%%%%%%%%%%%%%%%%%%%%%%%%%%%%%%%%%%%
%
% File:   archaeopteryx.tex
% Author: Sudnya Diamos
%         Gregory Diamos
% Date:   Saturday September 3, 2011
% Brief:  The latex source file for the Archaeopteryx simulator paper.
% 
%%%%%%%%%%%%%%%%%%%%%%%%%%%%%%%%%%%%%%%%%%%%%%%%%%%%%%%%%%%%%%%%%%%%%%%%%%%%%%%%

%\documentclass[12pt]{report}
\documentclass[conference, 10pt]{IEEEtran}

%%%%%%%%%%%%%%%%%%%%%%%%%%%%%%%%%%%%%%%%%%%%%%%%%%%%%%%%%%%%%%%%%%%%%%%%%%%%%%%%
% Included Packages
\usepackage{cite} 
\usepackage[pdftex]{graphicx} 
\usepackage{url}
\usepackage{booktabs} 
\usepackage{setspace}
%%%%%%%%%%%%%%%%%%%%%%%%%%%%%%%%%%%%%%%%%%%%%%%%%%%%%%%%%%%%%%%%%%%%%%%%%%%%%%%%

%%%%%%%%%%%%%%%%%%%%%%%%%%%%%%%%%%%%%%%%%%%%%%%%%%%%%%%%%%%%%%%%%%%%%%%%%%%%%%%%
% Configure Document
\topmargin      0.0in
\headheight     0.0in
\headsep        0.0in
\oddsidemargin  0.0in
\evensidemargin 0.0in
\textheight     9.0in
\textwidth      6.5in
%%%%%%%%%%%%%%%%%%%%%%%%%%%%%%%%%%%%%%%%%%%%%%%%%%%%%%%%%%%%%%%%%%%%%%%%%%%%%%%%

%%%%%%%%%%%%%%%%%%%%%%%%%%%%%%%%%%%%%%%%%%%%%%%%%%%%%%%%%%%%%%%%%%%%%%%%%%%%%%%%
% Configure Packages
\graphicspath{{images/}} 
\DeclareGraphicsExtensions{.pdf,.jpeg,.png} 
\pagenumbering{arabic}
%%%%%%%%%%%%%%%%%%%%%%%%%%%%%%%%%%%%%%%%%%%%%%%%%%%%%%%%%%%%%%%%%%%%%%%%%%%%%%%%

%%%%%%%%%%%%%%%%%%%%%%%%%%%%%%%%%%%%%%%%%%%%%%%%%%%%%%%%%%%%%%%%%%%%%%%%%%%%%%%%
%% New Commands
%%%%%%%%%%%%%%%%%%%%%%%%%%%%%%%%%%%%%%%%%%%%%%%%%%%%%%%%%%%%%%%%%%%%%%%%%%%%%%%%

%%%%%%%%%%%%%%%%%%%%%%%%%%%%%%%%%%%%%%%%%%%%%%%%%%%%%%%%%%%%%%%%%%%%%%%%%%%%%%%%
% High Level Organization (See individual sections for details) - 8000 words
% 
% Chapter 1) - Introduction                            - 1000 words
% Chapter 2) - Executive Summary (emphasis on speedup) - 750  words
% Chapter 3) - Framework Design                        - 1750 words
% Chapter 4) - Simulator Core Design                   - 3000 words
% Chapter 5) - Experimental Evaluation                 - 1000 words
% Chapter 6) - Related Work                            - 500  words
% Chapter 7) - Conclusions                             - 500  words
%
%%%%%%%%%%%%%%%%%%%%%%%%%%%%%%%%%%%%%%%%%%%%%%%%%%%%%%%%%%%%%%%%%%%%%%%%%%%%%%%%

\begin{document} 

%%%%%%%%%%%%%%%%%%%%%%%%%%%%%%%%%%%%%%%%%%%%%%%%%%%%%%%%%%%%%%%%%%%%%%%%%%%%%%%%
% Title and Authors
\title{Archaeopteryx\\
Simulating Future Architectures on GPUs}

\author{Sudnya Diamos and Gregory Diamos  \\
No Affiliation \\
{\small mailsudnya@gmail.edu, gregory.diamos@gatech.edu}}
\date{\today}

\maketitle
%%%%%%%%%%%%%%%%%%%%%%%%%%%%%%%%%%%%%%%%%%%%%%%%%%%%%%%%%%%%%%%%%%%%%%%%%%%%%%%%

%%%%%%%%%%%%%%%%%%%%%%%%%%%%%%%%%%%%%%%%%%%%%%%%%%%%%%%%%%%%%%%%%%%%%%%%%%%%%%%%
% Section 1 - 1000 words
\section{Introduction}

% Survey of the space
The transition to many core computing has coincided with the growth of
data parallel computation and the evolution of graphics processing
units (GPUs) from special purpose devices to programmable cores. 
The emergence of low cost programmable GPU computing substrates from
NVIDIA, Intel, AMD, and ARM have made data parallel architectures commodity
from embedded systems through large scale clusters such as the
Tsubame~\cite{ref:tsubame} and Keeneland systems~\cite{ref:keeneland}
hosting thousands of GPU chips.

The dominant programming systems involve the use of
bulk-synchronous-parallel programming models~\cite{ref:bulk-synchronous}
embodied by languages such as CUDA, OpenCL, and C++-AMP.  These data-parallel
languages implement \textit{single instruction stream multiple thread} (SIMT)
models of computation that specify a large number of
data-parallel threads that can be readily exploited by hardware
multi-threading and \textit{single-instruction-multiple-data} (SIMD)
cores. 

In contrast to many-core processors based on out-of-order CPU cores,
GPU architectures are designed to exploit the massive data-parallelism of
bulk-synchronous programs. Performance is maximized for regular computations
where hardware can use SIMD pipelines and bulk data transfers to exploit
control and data locality among threads.  However, current designs suffer
from steep performance cliffs when executing programs with irregular control
flow and data access patterns.  
%They are also 
%from branch divergence (when threads take
%different paths through the program) and memory divergence (when threads perform
%scatter or gather memory accesses).  
They are also beginning to introduce new
programming hazards as they move to \textit{non-uniform memory access} (NUMA)
organizations to reduce on-chip network overheads.  

These performance hazards complicate GPU programming. In fact, sequential
algorithms mapped to single core CPUs are still competitive with GPUs for many
application domains (e.g. database queries, loss-less compression,
lexing/parsing, or graph analytics) despite a 300x (and exponentially growing)
difference in peak throughput.  Two of the most important problems in computing
today involve designing architectures with more gradual performance cliffs
(without sacrificing parallel efficiency), and
designing bulk-synchronous algorithms for irregular applications.  

% attention grabber
Unfortunately, the tools that have been traditionally used to explore solutions
to these problems (architecture simulators and analytical models) fall into the
category of applications \textit{without} efficient parallel implementations. 
This has created a gap between parallel processor and simulator performance that
is widening with each successive generation, limiting the ability of architects
to explore the design space of possible optimizations and the ability of
application designers to evaluate the impact of algorithms on future
architectures.  

% Functional modeling
This is even more troubling for functional verification.  Although most bugs are
still caught with relatively short directed tests~\cite{ref:bug-distributions},
a critical class of bugs only occurs during complex interactions between
multiple components.  These bugs are typically discovered through application
stress testing or random program execution.   As application complexity
continues to grow with processor capability, there is a rising concern that
simulator performance will limit application and random test coverage,
especially for applications with multiple phases.  

% Why are we doing this?
It would be desirable to leverage the tremendous growth in computing potential
provided by parallel processors for microarchitecture simulation.  However,
the sequential operation and tight dependency loops in high performance 
processor pipelines quickly dash the hopes of straightforward solutions.
Exploiting fine-grained data parallelism and expressing the operation of a
processor in terms of hierarchical bulk data transfers requires a ground-up
approach.

% Our approach
In this paper, we assert that the parallel and hierarchical organization of
microarchitecture structures used in modern processors provide a natural
basis for simulation using data-parallel algorithms.  We test this assertion by
implementing a composable functional simulator for a new processor architecture,
and then augment it with performance models for key micro-architecture
structures.  We exploit fine-grained data parallelism by using hundreds of
threads to simulate a single core.  We show how basic parallel algorithm
building blocks like reductions, prefix sums, sorts, and histograms can be used
to simluate instruction fetch units, register files, thread schedulers,
functional units, and caches.  Across X applications, we show that the accuracy
of the simulation is comparable to state-of-the-art sequential simulators, and
that the performance of the simulator scales across three generations of GPUs.
This paper describes the first data-parallel processor simulator that can 
leverage the technology scaling and microarchitecture enhancements of the
previous generation of GPU hardware to simulate the next.

% Contributions
Specifically, this paper makes the following contributions:

\begin{itemize}
	\item It describes the design of the first processor simulator implemented
		completely in the CUDA "C" language.

	\item It introduces a design methodology for simulating parallel processors
		that achieves near peak utilization of modern GPU accelerators and
		good scaling across three generations of GPUs.
		
	\item It shows that microarchitecture details, such as scratchpad sizes,
		hardware thread schedulers, register file organizations, cache
		hierarchies, and core pipelines can be simulated quickly as long as
		the hierarchical organization of the machine is maintained.
	
	\item It compares the results and performance of our parallel simulator with
		two existing GPU simulators: GPGPU-Sim, and the Ocelot PTX Emulator.  
		We demonstrate a Nx speedup.
\end{itemize}

%%%%%%%%%%%%%%%%%%%%%%%%%%%%%%%%%%%%%%%%%%%%%%%%%%%%%%%%%%%%%%%%%%%%%%%%%%%%%%%%

%%%%%%%%%%%%%%%%%%%%%%%%%%%%%%%%%%%%%%%%%%%%%%%%%%%%%%%%%%%%%%%%%%%%%%%%%%%%%%%%
% Section 2 - 750 words
\section{Executive Summary}
%%%%%%%%%%%%%%%%%%%%%%%%%%%%%%%%%%%%%%%%%%%%%%%%%%%%%%%%%%%%%%%%%%%%%%%%%%%%%%%%

%%%%%%%%%%%%%%%%%%%%%%%%%%%%%%%%%%%%%%%%%%%%%%%%%%%%%%%%%%%%%%%%%%%%%%%%%%%%%%%%
% Section 3 - 1750 words
\section{Framework Design}
%%%%%%%%%%%%%%%%%%%%%%%%%%%%%%%%%%%%%%%%%%%%%%%%%%%%%%%%%%%%%%%%%%%%%%%%%%%%%%%%

%%%%%%%%%%%%%%%%%%%%%%%%%%%%%%%%%%%%%%%%%%%%%%%%%%%%%%%%%%%%%%%%%%%%%%%%%%%%%%%%
% Section 4 - 3000 words
\section{Simulator Core Design}
%%%%%%%%%%%%%%%%%%%%%%%%%%%%%%%%%%%%%%%%%%%%%%%%%%%%%%%%%%%%%%%%%%%%%%%%%%%%%%%%

%%%%%%%%%%%%%%%%%%%%%%%%%%%%%%%%%%%%%%%%%%%%%%%%%%%%%%%%%%%%%%%%%%%%%%%%%%%%%%%%
% Section 5 1000 words
\section{Experimental Evaluation}
%%%%%%%%%%%%%%%%%%%%%%%%%%%%%%%%%%%%%%%%%%%%%%%%%%%%%%%%%%%%%%%%%%%%%%%%%%%%%%%%

%%%%%%%%%%%%%%%%%%%%%%%%%%%%%%%%%%%%%%%%%%%%%%%%%%%%%%%%%%%%%%%%%%%%%%%%%%%%%%%%
% Section 6 - 500 words
\section{Related Work}

% PTL-Sim

% RAMP

% GPGPU-SIM, Barra, GPU-Ocelot
\cite{ref:ocelot-pact}

% 

%%%%%%%%%%%%%%%%%%%%%%%%%%%%%%%%%%%%%%%%%%%%%%%%%%%%%%%%%%%%%%%%%%%%%%%%%%%%%%%%

%%%%%%%%%%%%%%%%%%%%%%%%%%%%%%%%%%%%%%%%%%%%%%%%%%%%%%%%%%%%%%%%%%%%%%%%%%%%%%%%
% Section 7 - 500 words
\section{Conclusion}
%%%%%%%%%%%%%%%%%%%%%%%%%%%%%%%%%%%%%%%%%%%%%%%%%%%%%%%%%%%%%%%%%%%%%%%%%%%%%%%%

%%%%%%%%%%%%%%%%%%%%%%%%%%%%%%%%%%%%%%%%%%%%%%%%%%%%%%%%%%%%%%%%%%%%%%%%%%%%%%%%
% Bibliography
\bibliographystyle{IEEEtran}
\bibliography{archaeopteryx}
%%%%%%%%%%%%%%%%%%%%%%%%%%%%%%%%%%%%%%%%%%%%%%%%%%%%%%%%%%%%%%%%%%%%%%%%%%%%%%%%

\end{document}

